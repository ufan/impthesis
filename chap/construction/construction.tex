\chapter{塑闪阵列探测器的建造与质量控制}
\label{ch:construction}
\begin{enumerate}
	\item 只对探测器部分的建造过程进行简单介绍,不涉及机械结构(支撑功能模块)和电路部分(前段电子学和高压模块)。
	\item 探测器部分的建造可分为:PMT组件的装配、单元条的装配以及探测器整体组装。其中单元条的装配在张永杰论文中已有详细介绍,这里略过。
	\item 特殊的空间使用环境对
\end{enumerate}

\section{空间环境对探测器建造的特殊要求}

\section{PMT的筛选}
\subsection{筛选方案}
工厂参数。
% 暗电流
PMT的暗电流是指在完全黑暗且没有入射光的条件下,在光电倍增管内部流动的微小电流。
由于暗电流会使得探测器的基线展宽、噪声变大,严重的情况下甚至降低探测器的能量分辨率,因此暗电流越小越好。
光电倍增管的光阴极材料是引起暗电流的主要因素。
PSD采用的R4443型号(即MOD2)使用低噪声的碱金属材料作为光阴极,从而极大地抑制了暗电流的大小。
根据Hamamatsu提供的出厂测试信息,我们发现所有管子的暗电流都小于PSD的要求。
% 增益
R4443的Dy8通道用于覆盖PSD动态范围的低端部分,主要用于$e/\gamma$;R4443的Dy5通道用于覆盖PSD动态范围的高端部分,主要用于相对论重离子的电荷测量。
我们希望Dy8的增益尽量高,使其测得的MIP信号尽量与基线噪声相分离,从而降低$e/\gamma$误判率。
另一方面,PSD对整体的动态范围有严格的区间要求($\SI{0.1}{MIPs}\sim\SI{1400}{MIPs}$),过高的Dy8增益会压低Dy5的增益(即提高了Dy58比值),从而降低Dy5对重离子电荷的分辨能力。
因此
\subsection{参考单元模块的MIPs响应}
\subsection{与塑闪单元条的匹配}

\section{PMT组件的装配}
\label{sec:construction:pmt_assembly}
\subsection{结构简介}
\subsection{生产流程}
\begin{enumerate}
	\item PMT的base焊接,信号线+信号头。
	\item 检测:电压、电容。
	\item 灌胶。
	\item 高低温循环。
	\item 真空高压测试。
	\item PMT测试平台测试
\end{enumerate}

\section{探测器整体组装}
\label{sec:construction:psd_assembly}
\begin{enumerate}
	\item 布单元条。
	\item PMT组件加套筒和播磨合金
	\item 上PMT。
	\item 宇宙线测试。
	\item 上高压扇出板
	\item 上FEE盒子(焊接定位用),上FEE转接板
	\item 上高压接头
	\item 正式上FEE盒子
	\item 上FEE电路板
	\item 上顶板
\end{enumerate}

\section{环境实验}
