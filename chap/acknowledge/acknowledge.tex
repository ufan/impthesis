% vim:ts=4:sw=4
% Copyright (c) 2014 Casper Ti. Vector
% Public domain.

\chapter*{\hypertarget{ack}{致~~谢}}
\thispagestyle{empty} %覆盖\chapter*{}中的调用的\thispagestyle{plain},因为此页不需要页眉页脚
\pdfbookmark[1]{致谢}{ack}
% 中文测试文字。

忙碌而充实的博士生涯即将结束,谨借此机会,向帮助我和鼓励我的所有人献上最诚挚的谢意。

首先,衷心地感谢我的导师孙志宇研究员。
在硕博六年的研究过程中,孙老师无论是在学习上还是生活上,都给予了我极大的关怀与支持。
正是在您的引导下,我在理论知识、实验技能、论文写作等方面受到了比较综合的训练,并逐步学会了做科研的方法。
同时,孙老师深厚的学术功底、严谨的治学态度以及勤恳的敬业精神,无一不在潜移默化中感染着我,并激励和鞭策我不 断学习、努力进取。

特别感谢我的师兄余玉洪副研究员,他不仅是我在探测器领域的启蒙老师,也给了我很多生活、学习以及事业上中肯的建议。
在将近三年的暗物质粒子探测卫星工程研制阶段,余师兄直面挑战,克服困难,体现了一个科研工作者吃苦耐劳、勇往直前的精神。
这种精神感染了我,教会我脚踏实地做事,诚诚恳恳对人,是我人生道路上一笔宝贵的财富。

感谢次级束物理研究组唐述文、王世陶、章学恒、岳柯副研究员以及科学仪器中心的段利敏研究员,他们解答我在科研工作中碰到的问题,并无私地分享他们成功的经验。

感谢暗物质粒子探测卫星塑闪阵列分系统研制团队的所有老师、同学以及工作人员,是大家的通力合作使得我们圆满完成了塑闪阵列探测器的研制工作。特别地,我要感谢次级束研究组的陈俊岭、方芳、张永杰、王兆民,核电子学组的苏弘研究员、赵红赟、孔杰、杨海波、张惊蛰,磁铁组的杨雅清研究员、杨鹏,辐射材料组的刘杰研究员。

最后,感谢我的父母对我的鼓励和支持,你们一直是我追求学业的动力。

\par\hfill
\par\hfill\textbf{周~勇\hspace{10mm}}
\par\hfill 2016年4月12日