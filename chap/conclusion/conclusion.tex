% vim:ts=4:sw=4
% Copyright (c) 2014 Casper Ti. Vector
% Public domain.

\chapter{总结与展望}
\label{ch:conclusion}

\section{总结}

DAMPE卫星自2015年12月17日发射并进入轨道后,已经完成了三个月的在轨测试阶段,并正式进入到巡天观测模式。
在轨测试期间,PSD探测器整体工作正常,各项性能指标符合设计要求并非常稳定,从而验证了PSD的设计合理性,并说明PSD建造过程的质量控制过关。

本论文完整展现了PSD从整体设计、组件原理验证、探测器建造以及宇宙线标定的整个研制过程。
其中,论文的主要工作集中在细致研究了影响PSD探测性能的几个关键技术,完成了PSD的建造,并对PSD整体进行了地面宇宙线标定。
另外,论文工作过程中还完成了两套大型的探测器辅助测试平台的设计和研制,这些测试平台为PSD的研制成功提供了坚实的基础。
下面,对本论文取得的主要成果进行一个小结:
\begin{enumerate}
	\item PSD关键技术的细致研究:
	\begin{itemize}
		\item PSD需要覆盖接近4个量级的动态范围区间,这是PSD探测器主体功能模块的主要设计难点。本论文完成了基于Dy5和Dy8双打拿极信号引出的大动态范围读出方案的设计与实验验证,结果显示该设计方案能够满足PSD的动态范围要,并最终应用到PSD的正样飞行件上。
		\item PMT的工作状态直接关系到PSD探测单元模块的性能,大规模的探测器系统中往往需要对大量的PMT进行测试并进行筛选,对于PSD来说此项筛选流程更加严格。本论文对570支候选R4443裸管进行了测试,使用相对增益测量方法得到了它们的增益特性曲线,同时也得到了绝对的Dy58比值增益特性曲线。根据PSD的特殊要求,本论文确定了严格的筛选条件,并根据PMT的测试结果进一步从570支PMT中挑选出164支安装到PSD上。
	\end{itemize}
	\item PSD的建造:探测器的建造相对来讲是一个较为琐碎的过程,然后PSD的建造过程中涉及了大量的工艺和质量测试程序,使得它可以单独成为一个系统,并直接决定PSD最终的探测器性能。本论文的工作完整参与了PSD的整个建造流程。
	\item PSD的宇宙线测试:本论文对装配完成的PSD进行了长度15天的宇宙线测试,并从中提取除了PSD的性能参数,发现PSD的实际性能达到并好于设计指标;另外,本文还从宇宙线标定数据中提取出了PSD的第一批刻度数据,这些数据对于研究PSD的能量重建和模拟具有重要价值。
	\item 大型辅助测试平台的研制:
	\begin{itemize}
		\item PMT批量测试平台是本论文为PSD的R4443测试专门设计和建造的,它最多能够同时测量25支PMT,并具有光阴极扫描功能。考虑到在日常的探测器研制过程经常会有PMT测试工作,本论文对该平台采用模块化设计,其硬件组件和软件组件都很容易更换或更新以适应新的应用需求。
		\item 地面宇宙线标定测试平台是本论文为PSD的宇宙线标定专门设计和建造的,并用它完成了PSD的初次宇宙线刻度。同样的,该测试平台也适合在其它探测器的研制过程中使用。
	\end{itemize}
\end{enumerate}

\section{展望}
PSD的研制工作随着DAMPE的发射成功以及在轨测试的顺利完成而圆满结束。
现在,DAMPE已经进入正式的物理观测模式,每天都是大量的数据下传,同时也意味着对PSD的研究中心将转向物理量的精细重建上,主要有:
\begin{enumerate}
	\item 完成PSD每个探测单元模块的精细能量重建,包括光衰减效应的修正,两端的能量信息提取及综合等。
	\item 完成PSD探测器整体的能量重建,包括击中交叠单元模块的事件处理,量能器反照粒子的干扰去除等。
	\item 完成PSD的Geant4模拟中的数字化工作。
	\item 实现PSD的$e/\gamma$分辨算法,这需要利用STK的重建径迹以及BGO的簇射形状重建。
	\item 完成PSD的在轨标定工作,包括MIPs响应,Dy58比值,电子学刻度,基线以及这些量随环境参数的变化等。
\end{enumerate}