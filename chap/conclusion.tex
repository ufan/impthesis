% vim:ts=4:sw=4
% Copyright (c) 2014 Casper Ti. Vector
% Public domain.

\chapter{结论}
% 中文测试文字。
\section{中文测试}
\pkuthssffaq

\section{通知一}

各学院、各位研究生:

为维护研究生考试的公平、公正,严肃考试纪律,保证本学期期末考试工作的顺利进行,现将有关事项通知如下。

一、考试时间及考场安排

研究生期末考试定于2015年1月5日至1月16日进行。公共课由研究生院及开课学院安排,专业课由各学院负责组织安排与实施。公共课考试的具体安排见附件一、附件二及附件三。

二、考试纪律

研究生参加考试必须严格遵守考场纪律,具体要求见附件四。对违反考试纪律、不服从监考教师管理的违纪、作弊考生,将根据情节轻重参照《兰州大学学生违反考试纪律处理办法》的有关规定予以处理。

三、有关要求

各学院要高度重视考试工作,严格组织考务及有关工作,组织我校正式教职工担任监考,并对监考人员严格要求,加强培训。对研究生加强考试纪律要求,严肃考风考纪。

各学院要成立由主管院长负责的考试工作小组,安排对专业课考试的巡考工作。研究生院将组织教学督导专家及工作人员开展巡考工作。

各学院务必通知研究生携带证件按时参加考试。研究生须认真阅读考场规则及考试安排,按要求时间地点准时参加考试,以免耽误考试。

\subsection{再次通知}
各学院、各位研究生:

为维护研究生考试的公平、公正,严肃考试纪律,保证本学期期末考试工作的顺利进行,现将有关事项通知如下。

一、考试时间及考场安排

研究生期末考试定于2015年1月5日至1月16日进行。公共课由研究生院及开课学院安排,专业课由各学院负责组织安排与实施。公共课考试的具体安排见附件一、附件二及附件三。

二、考试纪律

研究生参加考试必须严格遵守考场纪律,具体要求见附件四。对违反考试纪律、不服从监考教师管理的违纪、作弊考生,将根据情节轻重参照《兰州大学学生违反考试纪律处理办法》的有关规定予以处理。

三、有关要求

各学院要高度重视考试工作,严格组织考务及有关工作,组织我校正式教职工担任监考,并对监考人员严格要求,加强培训。对研究生加强考试纪律要求,严肃考风考纪。

各学院要成立由主管院长负责的考试工作小组,安排对专业课考试的巡考工作。研究生院将组织教学督导专家及工作人员开展巡考工作。

\section{通知二}
各学院务必通知研究生携带证件按时参加考试。研究生须认真阅读考场规则及考试安排,按要求时间地点准时参加考试,以免耽误考试。

为切实提高我校研究生尤其是博士研究生的科技论文英文写作水平,学校将按学科开设研究生科技论文英文写作公共选修课。本学期,研究生院特邀请澳大利亚阿德莱德大学Dr. Margaret Cargill授课。其中,博士研究生培训班在部分专业进行试点,具体事宜另行通知。现就青年教师培训班相关事宜通知如下:

1.培训的主要内容为科技论文的结构、文章的科学逻辑、英文写作及审稿要求、文稿的投递、评审与修改等。

2.培训范围为我校各学科青年教师20-30名,培训结束经考核合格后发放结业证书。参加培训者要承担本学科博士研究生科技论文英文写作课程的教学任务。

3.培训为英语授课,请各学院推荐一名青年教师参加培训,外语学院推荐三名。要求被推荐者具有较高的英语水平,能熟练运用英语进行听、说、读、写,有国外留学经历者优先。

4.请各学院推荐教师填写“科技论文英文写作培训班登记表”(附件1),并于5月4日前交至研究生院培养处(贵勤楼531室)。

5. 授课时间为5月18日-22日,授课地点另行通知。同时请参加培训的老师自行准备教材“Writing Scientific Research Articles-strategy and steps ( Margaret Cargill Patrick O’Connor 著,Wiley-Blackwell出版社2009)”(可自网上下载) 。

请各学院认真遴选推荐青年教师参加培训,并按时报送登记表,未尽事宜请与研究生院培养处联系。

联系人:吕伟东

联系电话:8912660

